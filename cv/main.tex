% ========================================================
% This document is a customizable CV/Resume template built using LaTeX.
% The template is designed for easy customization and clear structure.
%
% Author: Matthew DeVerna (www.matthewdeverna.com)
% Date: 2024
% Design: Cased on Hause Lin's CV (hauselin.com)
% 
% Project Overview:
% -----------------
% This LaTeX document is designed to help you create a professional CV or 
% resume with ease. It uses as little fancy LaTex functionality or custom functions as possible to maximize its longterm durability and flexibility.
% The document is structured into multiple sections, each loaded from 
% separate subfiles for modularity and ease of maintenance. 
%
% Key Features:
% -------------
% - Customizable sections: Education, Research Experience, Awards, Publications, etc.
% - Bookmarks in the PDF for easy navigation
% - Styled bibliography with BibLaTeX
% - Hyperlinked email and website
% - FontAwesome icons for additional styling
%
% Getting Started:
% ----------------
% 1. Customize your personal information by modifying the \mytitle command.
% 2. Add your content to the respective subfiles (e.g., education.tex, exp_research.tex).
% 3. Update the bibliography file (ref.bib) with your publications and categorize them with keywords.
%
% Important Notes:
% ----------------
% - This main file includes the overall structure and settings. 
% - Each section has its own detailed instructions for further customization.
%
% ========================================================

\documentclass[11pt]{article} % Choose the document class and font size
\usepackage[margin=1in]{geometry} % Page layout settings

% Set the citation style
\usepackage[
    backend=biber,      % Specifies the backend to be used by BibLaTeX for processing the bibliography. 'biber' is the default backend.
    maxnames=20,        % Limits the maximum number of author names to display before abbreviating with "et al."
    style=nature,       % Sets the citation style to 'nature,' which is commonly used in scientific papers.
    sorting=ydnt,       % Specifies the sorting order of entries in the bibliography:
                        % y - year (descending)
                        % d - descending order
                        % n - name
                        % t - title
    defernumbers=true,  % Delays the assignment of citation numbers until the end of the document, allowing for the correct order of citations within each bibliography section.
]{biblatex}
\addbibresource{ref.bib} % Adds the bibliography resource file 'ref.bib' containing all the references.

% Allows columns that stretch across pages
\usepackage{longtable}

% Table functionality and beautification (not strictly needed)
\usepackage{bookmark}

% Use icons, if you want.
% All available icons: http://mirrors.ibiblio.org/CTAN/fonts/fontawesome5/doc/fontawesome5.pdf
\usepackage{fontawesome}

% Allows font justification control (needed for clean pub-list formatting)
\usepackage{ragged2e}

% For underlining with line breaks
\usepackage{soul} 

% All fonts: https://tug.org/FontCatalogue/
\usepackage{kpfonts} % More professional font
% \usepackage[default]{sourcecodepro} % Code-like font
\usepackage[T1]{fontenc}

% Control hyperlinks and colors
% CUSTOM COLORS INCLUDED DIRECTLY AFTER \begin{document}
\usepackage{xcolor}
\usepackage{hyperref}
\hypersetup{
    colorlinks=true,        % Enable colored links
    breaklinks=true,        % Allow links to break across lines
    linkcolor=cornflowerblue,    % Color of internal links
    urlcolor=cornflowerblue,     % Color of URL links
    anchorcolor=cornflowerblue,  % Color of anchors
    citecolor=cornflowerblue,    % Color of citations
    pdftitle={CV},    % Title of the PDF
    pdfauthor={Hector Kohler}, % Author of the PDF
    bookmarksopen=true,      % Open bookmarks panel at start
}

%%% CONVENIENCE FUNCTIONS GO HERE %%%
%%% ----------------------------- %%%
\newcommand{\mytitle}[4]{
  \begin{center}
    \Large\textbf{#1}\normalsize \\ % Name in large bold font
    \href{mailto:#2}{#2} \\ % Email with mailto: link
    \href{https://#3}{#3} \\ % Website with link
    #4 % Address
  \end{center}
}
%%% ----------------------------- %%%


\begin{document}
% Set custom colors here (imported directly after \begin{document})
% The below use HTML hex codes.
% More HTML hex codes: https://encycolorpedia.com/html
\definecolor{firebrick}{HTML}{b22222} 
\definecolor{darkslategrey}{HTML}{2f4f4f} 
\definecolor{cornflowerblue}{HTML}{00008b} 
\definecolor{mediumslateblue}{HTML}{7b68ee}  % Load custom colors from colors file
\mytitle{Hector Kohler}{hector.kohler@inria.fr}{https://kohlerhector.github.io/homepage/}{\href{https://github.com/KohlerHECTOR/}{github}} % Insert your custom title


% Ensure right side margin is not surpassed by bibliography and the right margin is aligned throughout
\RaggedRight


% These \pdfbookmark lines create bookmarks in the exported PDF document that display in the left pane.
% Value in [] sets the indentation level of the bookmark
\pdfbookmark[1]{Education}{}
\section*{Education}
% Add your educational background here!

% NOTE: If you want to remove the "Expected" footnote, you will want to remove:
% - Directly below: \renewcommand, \setcounter
% - In the table: \footnotemark in the left column
% - After the table: \footnotetext, \renewcommand, \setcounter

% Different numbers in "\setcounter{footnote}{0}" use different symbols
\renewcommand{\thefootnote}{\fnsymbol{footnote}}
\setcounter{footnote}{0}

\begin{longtable}[l]{@{}p{.125\textwidth} p{0.875\textwidth}}
    % Use custom symbol footnote for "expected"
    2022-2025\footnotemark & \textbf{Ph.D.}, Interpretable and Reproducible Reinforcement Learning (RL), Universit\'e de Lille \\

    2020-22 & \textbf{M.Sc.}, Artificial Intelligence \textit{Excellence Track}, Sorbonne Universit\'e \\

    2016-20 & \textbf{B.Sc.}, Computer Science \& Mathematics, University of Manchester
% \end{tabularx}
\end{longtable}


% Add text for the custom footnote
\footnotetext[1]{Expected.}

% Restore the default footnote numbering
\renewcommand{\thefootnote}{\arabic{footnote}}
\setcounter{footnote}{1}


\pdfbookmark[1]{Research Experience}{exp_research}
\section*{Research Experience}
\label{exp_research}
% List your research experience here

\begin{longtable}[l]{@{}p{.125\textwidth} p{0.875\textwidth}}
    2025 & Visiting Researcher designing new RL algorithms and evaluation tools, \href{http://rlai.ualberta.ca/}{RLAI Lab}, University of Alberta, (Advisor: \href{https://cifar.ca/fr/biographie/adam-white/#topskipToContent}{Dr. A. White}) \\

    2021 & Research Assistant working on unmanned underwater drone control, \href{https://www.ensta-bretagne.fr/fr/recherche-en-technologies-de-linformation-et-de-la-communication-lab-sticc}{Lab-STICC}, ENSTA, (Advisors: \href{https://www.ensta-bretagne.fr/clement/}{Pr. B. Clement} and \href{https://www.ensta-bretagne.fr/fr/annuaire/gilles.le.chenadec}{G. Le Chenadec}) \\

    2020-2021 & Research Assistant working on Reinforcement Learning, \href{https://www.isir.upmc.fr/}{ISIR}, Sorbonne Universit\'e, (Advisor: \href{https://www.isir.upmc.fr/personnel/sigaud/}{Pr. O. Sigaud}) \\

\end{longtable}

\pdfbookmark[1]{Awards \& Honors}{awards}
\section*{Awards \& Honors}
\label{awards}
% Include awards, honors, grant funding, etc. here

\begin{longtable}[l]{@{}p{.125\textwidth} p{0.875\textwidth}}
    % Use custom symbol footnote for "expected"
    2025 & \href{https://graduate-programmes.univ-lille.fr/en/international-mobility-grant}{Universit\'e de Lille International Mobility Grant} (1650 €) \\

    2022 & \href{https://anr.fr/Project-ANR-20-THIA-0014}{AI\_PhD\@ Lille} (Competitive admission to fully funded Ph.D. program) \\
   
    2022 & Sorbonne Universit\'e Computer Science Excellence diploma (\~ 10/400 students) \\
       
\end{longtable}



\pdfbookmark[1]{Publications}{pubs}
\section*{Publications}
\label{pubs}

% Add equal contribution dagger
\vspace{-.75em}
\small
\faGoogle~\href{https://scholar.google.com/citations?user=aSO7bZ0AAAAJ&hl=en}{Google Scholar}\\
\normalsize


\pdfbookmark[2]{Journal Articles}{journal-article}
\subsection*{Journal Articles}
\label{journal-article}
\newrefcontext[labelprefix=J] % Will prefix bibliography numbers with this letter
\nocite{*}% Ensures publications which are not cited in the document are included in the above sections
\printbibliography[
    type=article, % Only include @article ref.bib items
    heading=none, % Do not include header. Gives us more control.
    resetnumbers=true, % Start item counter from zero
    keyword=J % Include items in ref.bib with keyword={J}
]

\pdfbookmark[2]{Peer-reviewed Conference Proceedings}{conferences}
\subsection*{Peer-reviewed Conference Proceedings}
\label{conferences}
\newrefcontext[labelprefix=C]
\printbibliography[type=inproceedings,heading=none,resetnumbers=true,keyword=C]

\pdfbookmark[2]{Posters}{posters}
\subsection*{Posters}
\label{posters}
\newrefcontext[labelprefix=P]        
\printbibliography[type=misc,heading=none,resetnumbers=true,keyword=P]


\pdfbookmark[1]{Tools \& Software}{tools}
\section*{Tools \& Software}
\label{tools}
\subsection*{Reinforcement Learning}
\begin{itemize}
    \item[] \href{https://rlberry-py.github.io/rlberry/about.html#contributors}{Core contributor to \texttt{rlberry}}: \texttt{PyTorch} implementation of deep reinforcement learning agents. Compatibility with \texttt{stable-baselines3}. Advanced training analysis with robust stats. Usage of git for development and continuous integration. Organized hackhatons.
    \item[] \href{https://github.com/KohlerHECTOR/interpretable-rl-zoo}{Maintainer of a set of fully-transparent control policies for games and robots}: inspired by the \texttt{stable-baselines3-zoo} and the \texttt{gymnasium} API.
    \item[] \href{https://github.com/TimotheeMathieu/adastop}{designed statistical test for the reliable and cheap comparison and benchmarking of RL.}
\end{itemize}


\subsection*{Supervised Learning}

\begin{itemize}
    \item[] \href{https://github.com/KohlerHECTOR/DPDTreeEstimator}{Maintainer of my own \texttt{scikit-learn} estimator}: I developped DPDT, a decision tree induction algorithm which is SOTA for tabular problems. 
\end{itemize}




% Include any additional details here
% \vspace{-.75em}
% \small
% $\dagger \rightarrow$ Equal contribution
% \normalsize


\pdfbookmark[1]{Teaching}{teaching}
\section*{Teaching}
\label{teaching}
\subsection*{Ecole Normale Sup\'erieure de Cachan (MVA)}
% 
\begin{longtable}[l]{@{}p{.125\textwidth} p{0.875\textwidth}}

    2024 & Assistant Instructor, Reinforcement Learning (6 hours)\\
    
\end{longtable}

\subsection*{Centrale Lille}
% 
\begin{longtable}[l]{@{}p{.125\textwidth} p{0.875\textwidth}}

    2024 & Assistant Instructor, Sequential Decision Making (6 hours)\\
    
\end{longtable}


\subsection*{Universit\'e de Lille}
% 
\begin{longtable}[l]{@{}p{.125\textwidth} p{0.875\textwidth}}

    2023-2024 & Assistant Instructor, Algorithms and Programming (60 hours)\\
    2022 & Assistant Instructor, Reinforcement Learning (24 hours)\\

\end{longtable}


\pdfbookmark[1]{Academic Advising}{advising}
\section*{Academic Advising}
\label{advising}
% Including academic advisement history here


% Remove \subsection{} lines and multiple tables if you only need one section!
\subsection*{Graduate}
\begin{longtable}[l]{@{}p{.125\textwidth} p{0.875\textwidth}}

    2023 & Veronika Shilova, Ecole Polytechnique, Entropy-regularized RL \\
    2023 & Quentin Dinel, Universit\'e de Lille, Image Segmentation \\
    2023 & Mathieu Medeng Essia, Centrale Lille, Imitation Learning \\
    2023 & Antonio Al-Makdissi, Centrale Lille, Object-centric Atari Learning \\
    2023 & Gabriele Maggioni, Centrale Lille, Model-Based Reinforcement Learning

\end{longtable}


\pdfbookmark[1]{Academic Service}{service}
\section*{Academic Service}
\label{service}


\subsection*{Conference Reviewer}
% 
\begin{longtable}[l]{@{}p{.125\textwidth} p{0.875\textwidth}}

    2025 & ICLR, RLC, AISTATS, ICML \\
    2024 & NeurIPS
    
% \end{tabularx}
\end{longtable}

\subsection*{Chairing}
% 
\begin{longtable}[l]{@{}p{.125\textwidth} p{0.875\textwidth}}

    2024 & Organized the \href{https://sites.google.com/view/interppol-workshop/home?authuser=0}{InterpPol} workshop at the first Reinforcement Learning Conference in Amherst (USA).
    
% \end{tabularx}
\end{longtable}

\subsection*{Responsabilities}
% 
\begin{longtable}[l]{@{}p{.125\textwidth} p{0.875\textwidth}}

    2023-2025 & Elected representative of Ph.D. students and Postdocs at the CRIStAL laboratory (Universit\'e de Lille).
    
% \end{tabularx}
\end{longtable}





\pdfbookmark[1]{Other Experience}{exp_other}
\section*{Other Experience}
\label{exp_other}
% Add other experience you don't know where to classify here

\begin{longtable}[l]{@{}p{.125\textwidth} p{0.875\textwidth}}
    
    2018 & Analyst Intern, \textbf{Barclays}, Radbroke (UK) \\
    2017 & AI Intern for A350 maintenance, \textbf{Airbus}, Madrid (Spain)

\end{longtable}


% Pretty ending with the date last updated
\centering
\rule{0.25\linewidth}{0.4pt}\\
\medskip
Last updated: \today

\end{document}
